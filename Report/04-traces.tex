\chapter{Generating Traces}
% \section{}
\subsection{Sancho}
To generate optimal gameplay we decided that the best approach was to use a previous winner of the GGP competition. Since the aim of the competition is to find the program that performs best at the set of games used in the IGGP problem we conclude that there is no better way to generate a comprehensive set of game traces. The winner of the 2014 GGP competition 'Sancho' was selected \footnote{http://ggp.stanford.edu/iggpc/winners.php accessed on 12/03/2020} since they are the most recent winner to have published their code \footnote{http://sanchoggp.github.io/sancho-ggp/ access 12/03/2020}. Some small modifications to the information logged by the game server are made but otherwise the code is unchanged.

The core of Sancho is the Monte Carlo tree search (MCTS) algorithm.

\subsubsection{Monte Carlo Tree Search}
Given a game state the basic MCTS will return the most promising next move. The algorithm achives this by simulating random playouts of the game many times. The technique was developed for computer Go but has since been applied to play a wide range of games effectively including board games and video games\cite{Silver/MCTS}\cite{Chaslot/MCTS}.

The use of random simulation to evaluate game states is a powerful tool. Information about the game such as heuristics for evaluating non terminal states are not needed at all, the rules of the game are enough on their own. In the case of the GGP problem this is ideal. The rules of the game are only revealed shortly before the game is played meaning deriving effective heuristics is a hard problem.

In our case all games being played are sequential, finite and discrete so we only need to consider MCTS for this case.
% say why it is good for this

% possibly put in a diagram of game tree
The MCTS is a tree search algorithm, the tree being searched is the game tree. A game tree being a tree made up of nodes representing states of the game and the children of each node being all the states that can be moved to from that state. The leaves of this tree are the terminal states. The search is a sequence of traversals of the game tree. A traversal is a path that starts at the root node and continues down until it reaches a node that has at least one unvisited child. One of these unvisited children are then chosen to be the start state for a simulation of the rest of the game. The simulation chooses random moves, playing the game out to a terminal state. The result of the simulation is propagated back from the node it started at all the way to the root node to update statistics attached to each node. These statistics are used to choose future paths to traverse so more promising moves are investigated more.

Exactly how the tree search algorithm chooses new nodes to simulate is where the complexity lies. The node chosen by traversal of the tree should strike a balance between exploiting promising nodes and exploring nodes with few simulations. The Upper Confidence Bound for trees algorithm was designed to do exactly this\cite{Kocsis/UCT}. The algorithm chooses a child node for each state based on the UTC formula. The formulae can be written as: \[UTC(v_i) = \frac{Q(v_i)}{N(v_i)} + c\sqrt{\frac{ln(N(v_i))}{N(v_i)}}\] Where we have:
\begin{itemize}
\item $v_i$ is node $v$ after move $i$
\item $Q(v)$ is the number of winning simulations that have taken place below it
\item $N(v)$ is the number of simulations that have taken place below it
\end{itemize}
The function is the sum of two components. $\frac{Q(v_i)}{N(v_i)}$ is called the exploitation component. It is a ratio of winning to loosing simulation that resulted from making this move. This encourages traversal of promising nodes that have a high win rate. However, if only this factor was used the algorithm would quickly find a winning node and only explore that hence we need a second component: the exploration factor. $c\sqrt{\frac{ln(N(v_i))}{N(v_i)}}$ favours nodes that have not yet been explored. The value $c$ is a constant that balances the two components, it can be adjusted depending on the use case.

Sancho makes a few modifications to MCTS\footnote{https://sanchoggp.blogspot.com/2014/05/what-is-sancho.html accessed on 15/03/2020}, the main one being adding of heuristics. The tree is also replaced with a more general graph which allows for transitions between lines of play without duplication of states. There are other modifications to increase efficiency.

In GGP matches a period of time before the match is given in which to do pre match calculations. Sancho uses this period to derive basic heuristics and optimise the value of $c$ in the formulae. A heuristic should take a game state or move and return a value based on how promising that state or move is in relation to the goal state. To take the heuristics into account when choosing the next state to explore each node (state) $v$ is seeded with a heuristic value on creation.

The identification of possible heuristics takes place in two stages. The first of these is static analysis of the game rules. This static analysis identifies possible heuristics that can be applied to the current game. These include things like piece capture: if certain rules indicate capture of a piece these can be selected against, or numeric quantity detection: a number in the state acts as a heuristic (like number of coins a player has). The second stage is simulation of the game. Many (possibly tens of thousands) of full simulations of the game are run. After the simulations are complete a correlation coefficient is calculated between each candidate heuristic's observed values and the eventual score achieved in the game. Those heuristics that show correlation above a fixed threshold are then enabled, and will be used during play.

%\subsection{Other aspects of Sancho}
Monte carlo simulations are used for a large number of the games available however Sancho identifies games that can be solved more efficiently in other ways. Any single player game in the GGP dataset can be analyised using puzzle solving techniques.
\subsubsection{Puzzle Solving}
A single player game in the GGP dataset is neciserally a deterministic game of complete information. This is due to the constraints of the Game Description Language, it is not possible to express anything more. This gives the constraint that any solution found in a playout of the game will always remain valid since the game is completely deterministic. Sancho identifies these single player games and attempts to derive huristics. Where an obviouse goal state exsists a distance metric such as the hamming distance between two states can be applied. A* search can then be used to find optimal solutions. For some games, such as eight puzzle the hamming distance is an admisable huristic so sancho finds an optimal solution.

Sancho also does static analysis of the game rules. It determins game predicates that if true mean the goal will never be true in any proceding state as well as predicates that garentee that the goal will be true in a future state. An example of this is a game such as untwisty corridor (the game is effecitvly a maze where if you immidiatly lose if you step off a 'safe' path). If the safe path is left then the goal can never be reached so any state not on the safe path is avoided.

\af{Talk about why this makes sancho good for this task}



\section{Generating the traces}
\ac{break this section down a bit}
To determine the influence of the quality of training data upon the ability of learners to solve the IGGP problem we initially generated the gameplay.

The general game playing community have a codebase that provides the basic functionality for hosting a match between GGP agents\footnote{https://github.com/ggp-org/ggp-base}. This codebase along with the GGP agent Sancho were used to generate the datasets. The codebase includes a very basic random GGP player that will, given a GDL game description, play a random legal move every turn. It uses the GDL description to generate all possible legal moves given the current game state then chooses one uniformly at random. To generate the random traces for a game of $x$ players $x$ random gamers were pitted against one another. The same method was used with Sancho instead of the random gamers to generate the high quality traces.

For each game written in GDL is represented by a set of ground atoms

Information about games is represented as sets of ground atoms. There exists a set $S$ that represents all atoms that can change from one state to the other. Examples of elements of this set $S$ might be the state of the board in \textit{tic tac toe} along with which player is to take their turn next:
\begin{verbatim}
( control noughts )
( cell 1 1 b ) ( cell 1 2 x ) ( cell 1 3 o )
( cell 2 1 x ) ( cell 2 2 o ) ( cell 2 3 o )
( cell 3 1 b ) ( cell 3 2 x ) ( cell 3 3 b )
\end{verbatim}
This set of state atoms includes all that can change throughout the game.

There is also the set of ground atoms representing the actions that can be taken $A$. The moves that each player makes are taken from this set. For a game such as \textit{eight puzzle} these would consist or the atoms \verb|( move i j )| for $0<i\leq 9$ and $0<j\leq 9$ which represent sliding the tile at $(i,j)$ into the space. For \textit{tic tac toe} they would be \verb|( mark i j )| for $0<i\leq 3$ and $0<j\leq 3$ to represent marking a square with the players symbol. To represent a player not making a move we also always have $noop \in A$.

We modified the GGP matchmaker to log the following information:
\begin{itemize}
	\item \textbf{Game state trace} - The sequence of game states: $states = (s_1,...,s_n)$ where each $s_i \subseteq S$ is the set of ground atoms true in the $i^{th}$ state.
	\item \textbf{Game roles} - The list of roles of each player in the game: $roles = (r_1,...,r_k)$ e.g. \textit{noughts} and \textit{crosses} in \textit{tic tac toe}
	\item \textbf{Move trace} - The sequence of moves made by each player after each state: $moves = ((m_{1,r_1},...,m_{1,r_k}),...,(m_{n,r_1},...,m_{n,r_k}))$ where $m_{i,r_j} \in A$ is the $i^{th}$ move of player $r_j$. In games where not all players move every turn then $m_{i,r_j}=noop$ shows that $r_j$ made no move
	\item \textbf{Legal move trace} - The sequence of the legal moves for each player in each state:  $legal = ((l_{1,r_1},...,l_{1,r_k}),...,(l_{n,r_1},...,l_{n,r_k}))$. $l_{i,r_j} \subseteq A$ is the set of possible moves for $r_j$ in state $s_i$
	\item \textbf{Goal value trace} - The sequence of goal value for each player on every state: $goals = ((g_{1,r_1},...,g_{1,r_k}),...,(g_{n,r_1},...,g_{n,r_k}))$. $g_{i,r_j} \in [0,100]$ represents how well player $r_j$ has achieved the goal in state $s_i$. For some games such as eight puzzle this is 0 on every state until the winning state where it becomes 100

\end{itemize}
This represents all the data needed from the match to generate IGGP tasks. In each state all the positive atoms in $S$ (that is the atoms ture in the current state) are arguments of the \texttt{true} predicate. For example if at the start of the game the first cell is blank on the board then we would have $\texttt{( true ( cell 1 1 b ) )} \in s_1$.

We define a function that transforms the sequences into four separate induction tasks. An induction task is the set of triples $(B,E^+,E^-)$. We first define a function trace that translates the sequences given in the log of the match into a set of pairs $(B,E^+)$ . We then define a function \textit{triple} which gives the the set of triples  $(B,E^+,E^-)$ from the pairs $(B,E^+)$.

We flatten the sequences of $(m_{i,r_1},...,m_{i,r_k})$ in \textit{moves}, $(l_{1,r_1},...,l_{1,r_k})$ in \textit{legals} and $(g_{1,r_1},...,g_{1,r_k})$ in \textit{goals} to a set that includes the role name as an extra argument of the predicate (the arity is increased by 1). For example if \[(m_{i,r_1},m_{i,r_2}) = \texttt{[( move 2 2 ), ( move 2 3 )]}\]
and $r_1 = black$ and $r_2 = white$ then it would be replaced by $m_i$
\[m_i = \texttt{\{( move black 2 2 ), ( move white 2 3 )\}}\]

\textit{moves} is now a sequence $(m_1,...,m_n)$. With \textit{legals} a similar intuative flattening procedure is applied
\begin{align*}
(l_{i,r_1},l_{i,r_2}) =\ [&\{\texttt{( move 2 2 ), ( move 1 1 ), ( move 1 3 )}\}, \\
&\{\texttt{( move 2 3 ), ( move 1 2 )}\}]
\end{align*}
and $r_1 = black$ and $r_2 = white$ then it would be replaced by $l_i$
\begin{align*}
l_i = \{&\texttt{( move black 2 2 ), ( move black 1 3 ),}\\ &\texttt{( move white 2 3 ), ( move white 1 2 )}\}
\end{align*}
Again we now have that legals is a sequence $(l_1,...,l_n)$. The same procedure is also applied to \textit{goals}.


The trace function is defined for \textit{legal}, \textit{goal}, \textit{next} and \textit{terminal}. We treat the output of zip as a set. We define $S[p/q]$ on a set $S$ for predicates $p$ and $q$ to be the substitution of all instances of $q$ for $p$.
\begin{align*}
&\Lambda_{legal} &&= trace_{legal}(states,legal) &&= \text{zip } states\ legal[\texttt{legal}/\texttt{true}] \\
&\Lambda_{goal} &&= trace_{goal}(states,goals) &&= \text{zip \textit{states goal}} \\
&\Lambda_{next} &&= trace_{next}(states,moves) &&= \text{zip \textit{states} } \text{(tail }states[\texttt{next}/\texttt{true}]) \\
&\Lambda_{terminal} &&= trace_{terminal}(states) &&= \text{zip \textit{states t}} \\
&&&\ \ \ \ \textbf{where  } t =\text{take } n\ (\text{repeat } \emptyset ) &&\text{++}\ \ [\{\texttt{( terminal )}\}] \\
&&&\ \ \ \ \textbf{and } n = (\text{length }states) - 1 &&
\end{align*}

The substitutions $[\texttt{legal}/\texttt{true}]$ and $[\texttt{next}/\texttt{true}]$ ensure that all positive examples in $E^+$ are in the relevant predicate. That is, we need all the \texttt{move} atoms in $legal$ to be inside the \texttt{legal} predicate:
\[\texttt{( legal ( move 1 1 ) )}\]
Since the predicates are already in the \texttt{true} predicate we only need to substitute one out for the other.

Some of the ILP systems being tested cannot work with function symbols of arity greater than 0. To allow them to operate we merge the function and their arguments into one predicate. For example \texttt{( legal ( move 1 1 ) )} becomes \verb|( legal_move 1 1 )| where \verb|legal_move| is a newly formed predicate.

Let $pred s p$ be the subset of atoms in $s$ that use the predicate $p$. For example when $s_{legal}\ S\ E^+$ is called with
\[X = \{\texttt{( move 1 1 ), ( move 1 2 ), ( move 2 1 ), ( move 2 2 )}, ...\}\]
and
\[E = \{ \texttt{( move 1 1 ), ( move 1 2 )}\}\]
then
\[s_{legal}\ X\ E = \{\texttt{( move 2 1 ), ( move 2 2 )}\}\]
To generate the triples we use the two functions $triples_1$ and $triples_2$.
\begin{align*}
tri&ple_1(\Lambda_p) = \text{map } f\ \Lambda \\
&\textbf{where } f\ (B,E^+) = (B,E^+,(pred\ S\ p) - E^+)\\
tri&ple_2(\Lambda_p) = \text{map } f\ \Lambda \\
&\textbf{where } f\ (B,E^+) = (B,E^+,(pred\ A\ p) - E^+)
\end{align*}
We generate the IGGP task with $triples_1$ and $triples_2$ as below:
\[\Delta = triples_1(\Lambda_{goal}) \cup triples_1(\Lambda_{terminal}) \cup triples_2(\Lambda_{next}) \cup triples_2(\Lambda_{legal})\]

