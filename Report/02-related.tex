\chapter{Related work}
%\ac{Use paragraphs}

\section{GGP}
General game playing is a framework for evaluating an agenet's general intelligence accross a wide range of tasks \cite{Cropper/IGGP}. The idea is that agents are able to accept declarative descriptions of arbitrary games at run time and are able to use such descriptions to play those games effectively. All the games are finite, discrete, deterministicmulti-player games of complete information. The games can vary in number of players, dimensions and complexy. For example games such as rock paper scissons have 0 dimensions and only 10 rules in the given GGP ruleset, more complex games such as checkers has 52 rules and is 2 dimensional. There are also single player games such as eight-puzzle or Fizz Buzz. The agents play games selected from an exsisting database which lists their descriptions described in the Game Description Language (GDL). GDL is a language based on first order logic described in section \ref{sec:GDL}. The list of games along with their descriptions is available online\cite{GGP-Website}. Matches in the GGP framework take place through an online framework called the Gamemaster. The agents connect to an online Game Manager (part of the Gamemaster) service which arbitrates indavidual matches. The connecting agents send several pieces of information to the Game Manger incliding a game that is already know by the manager and the required number of players. The match can then be started by pressing a start button, when pressed the agents will recive a \textit{Start} message including the role of the player (e.g. black or white in chess) and a description on the game in GDL. The Game Manger communicates all instructions to the players through HTTP\cite{Genesereth/GGPOverview}.

In 2005 an annual International General Game Playing Competition (IGGPC) was set up which still runs to this day. Each year hopeful participants pit GGP agents against one another to find the best one. The competitors take part in a series of rounds of increasing complexity. The agent that wins the most games in these rounds is the winner. The 2014 winner Sancho is used in this paper to generate optimal game traces for the IGGP task. The Game description language is used in the IGGP as an example of the rules to learn and the rules to use when generating examples. It is a type of logical programming language.

\subsection{GDL}\label{sec:GDL}
Game definition language is the formal language used in the GGP competition to specify the rules of the games.\cite{GDL_Spec} It is suited to specifying game rules because:
\begin{itemize}
\item It is a purely declarative language
\item It has restrictions to ensure that all questions of logical entailment are decidable
\item There are some reserved words, which tailor the language to the task of defining games
\end{itemize}
These descriptions define the games in terms of a set of true facts capturing the information needed to give the following predicates:
\begin{itemize}
\item The initial game state
\item The goal state
\item The terminal state
\end{itemize}
In addition, logical rules are used to describe the following:
\begin{itemize}
\item The legal moves for a given player and state
\item The next state for a given player state and move
\item The termination and goal conditions
\end{itemize}
In the IGGP problem the tasks are the rules for the goal, the next state, the legal states and the terminal state.


\section{Logical Programming}
Logic programming is a programming paradigm based on formal logic. Programs are made up of facts and rules. Rules are made up of two parts: the head and the body. They can be read as logical implications where the conjunction of all the elements in the body imply the head. The syntax is different for different logical programming languages but the head is usually written before the body. A fact is simply a rule without a body, that is, a statement that is taken as true. The compiler takes queries and returns weather they are true of false. If there are free variables in the query the compiler assigns them values for which the query is true. Logical programming is good for symbolic non-numeric computation. It is well suited to solving problems that involve well defined objects and relations between them, such as a GGP game.




\section{Prolog}
Prolog is one of the most popular and well used logical programming languages. The syntax in Prolog for rules and facts is fairly intuitive. A simple fact might be \mintinline{Prolog}{son(bob,alice).}. This tells us that the atom \mintinline{Prolog}{alice} relates the atom \mintinline{Prolog}{bob} in the \mintinline{Prolog}{son} relation. A rule is written \mintinline{Prolog}{parent(X,Y) :- son(Y,X)} which means if we have the relation son of Y and X then X relates Y in parent. The symbol :- is the same as reverse implication ($\Leftarrow$). We can query a program in the Prolog environment. If we typed in "\mintinline{Prolog}{parent(alice,bob).}" then it would return true because we have \mintinline{Prolog}{son(bob,alice)} and the rule which tells us that if X is a son of Y else then Y is a parent of X so \mintinline{Prolog}{alice} is a parent of \mintinline{Prolog}{bob}. Predicates can be conjoined with the comma "," (e.g. \mintinline{Prolog}{a(X,Y,Z) :- b(X,Y), c(Z,Y)}). Disjunction is expressed with the semicolon ";" (e.g. \mintinline{Prolog}{a(X,Y,Z) :- b(X,Y) ; c(Z,Y)}.

Prolog answers queries using a process know as proof seach and unification. The unification process is similar to logical unification, two terms unify if they are the same term or if they contain variables that can be instantiated with terms in such a way that the new terms are equal. For example the terms \mintinline{Prolog}{name(bob).} and \mintinline{Prolog}{name(X).} will unify with \mintinline{Prolog}{X = bob}. The Prolog ISO defines the Herbrand Algorithm for unification \cite{PrologISO}. A Prolog query is a set of goals. The Prolog System decides weather they are satisfiable or not. When a query is asked the of the Prolog system it executes something similar to the following algorithm. The exact implementations vary among Prolog systems however they are all roughly this algorithm.
\begin{algorithm}
\SetKwInOut{Input}{Input}
\SetKwProg{Fn}{Function}{}{}
\Input{A list of goals $GoalList = G_1,G_2,...,G_M$}
\Fn{\textbf{execute} (Program, GoalList, Success):}{
    \If{$empty(GoalList)$}{$Success \leftarrow true$}{
        \While{\textbf{not} empty($GoalList$)}{
            $\textit{Goal} \leftarrow head(GoalList)$\;
            $OtherGoals \leftarrow tail(GoalList)$\;
            $Satisfied \leftarrow false$\;
            \While{\textbf{not} Satisfied and there are more clauses in the program}{
                Let next clause in the program be $H \vdash B_1,...,B_n$\;
                Construct a variant of this clause $H' \vdash B_1',...,B_n'$\;
                $match(Goal,H',MatchOK,Instant)$\;
                \If{$MatchOK$}{
                    $NewGoals \leftarrow append([B_1',...,B_n'],OtherGoals)$\;
                    $NewGoals \leftarrow substitue(Instant,NewGoals)$\;
                    $execute(Program,NewGoals,$\textit{Satisfied}$)$\;
                }
            }
        $Success \leftarrow$ \textit{Satisfied};\;
        }
    }
}
\caption{Execute Prolog Goals}
\end{algorithm}

the program listing is wearched for a term to unify with. The listing is searched in the order it is written in. When Prolog finds a matching rule it then attempts to sequentially unify the terms of the body using the same method. If the rule has no body then the variables are assigned and the terms unify. If Prolog fails to unify two terms then it backtracks, assignes differnt values to \cite{Bratko}


\section{IGGP}
The IGGP problem uses the GDL game descriptions from the GGP framework to construct the IGGP dataset containing 50 games. The IGGP problem itself is described in section \ref{sec:LogicalSetting}. A machanism is also provided by Andrew Cropper et al. \cite{Cropper/IGGP} to turn the GGP games into new IGGP tasks. This mechanism has been modified and added to to generate optimal traces.


\section{ILP}\label{sec:ILP}
Inductive logic programming is a form of machine learning that uses logic programming to represent examples, background knowledge, and learned programs\cite{Cropper/EfficientLearning}. To learn the program is supplied with positive examples, negative examples and the background knowledge. In the general inductive setting we are provided with three languages.
\begin{itemize}
\item $\mathcal{L}_O$: the language of observations (positive and negative examples)
\item $\mathcal{L}_B$: the language of background knowledge
\item $\mathcal{L}_H$: the language of hypotheses
\end{itemize}
The general inductive problem is as follows: given a consistent set of examples or observations $O \subseteq \mathcal{L}_O$ and consistent background knowledge $B \subseteq \mathcal{L}_B$ find an hypothesis $H \in \mathcal{L}_H$ such that \[B \wedge H \vDash O\] \cite{Muggleton/ILP}
That is that the generated hypothesis and the background knowledge imply the positive examples and not imply the negative examples.

Example task (grandparent relation)

ILP systems generally regard ILP as a search problem. The search space is the set of well formed hypothesis. Often a set of inference rules are applied to the starting hypothesis, the new hypothesise are then pruned and expanded according to how often $B \wedge H \vDash O$ in the observations $O$. When all the observations are implied then a correct program has been found.



\section{Machine learning}
Machine learning theory contains a lot of work on the general setting of the problem we are considering, specifically, it covers how the relation between the training data and the testing data can affect the accuracy of the hypothesis as well as how the amount of training data can affect the result. In this paper

(in PAC) The cardinality of this example set must be at most a polynomial function of the size of the vocabulary used in constructing hypotheses.

More examples allows for longer programs therefore better programs?



\subsection{Metagol?}
?Should I add a section on how metagol works here as an example?


To evaluate ILP systems the learning tasks they are given need to be small enough that it is feasible for them to learn the programs, but also diverse enough that we test the full capability of the system. The Inductive general game playing framework provides this. To understand IGGP we first need to explain the General game playing framework.


