\begin{abstract}

\ac{GGP is ....}
General Game Playing (GGP) is a framework in which an artificial intelligence program is required to play a variety of games successfully. The framework includes repositories of game descriptions.
\ac{more, why is it interesting?}

\ac{IGGP is ....}
The Inductive General Game Playing (IGGP) problem challenges ILP systems to learn the rules of a game from observation of gameplay.
\ac{more, why is it interesting?}

\ac{However, a limitation of existing work is ....}
Existing work on IGGP has always assumed that the game player being observed makes random moves.
\ac{more, why is it interesting?}

\ac{To address this limitation we, .....}
To address this limitation, in this paper we analyse the effect of using optimal verses random gameplay traces as well as the effect of varying the number of traces in the training set.

The General Game Playing competition winner in 2014, Sancho, is used to generate optimal game traces for a large number of games and the systems, Metagol, Aleph and ILASP are trained and tested with combinations of optimal and random data.
\ac{Write with an active voice 'We use Sacho ...'}

\ac{Our results show ....}

\ac{The implication of this work is ....}



% When learning programs in Inductive Logic Programming (ILP) optimisations to the dataset used to train the systems can often be as effective as improvements to the systems themselves.


\end{abstract}